\documentclass[a4paper]{exam}

\usepackage{amsmath,amssymb, amsthm}
\usepackage{geometry}
\usepackage{graphicx}
\usepackage{hyperref}
\usepackage{titling}
\usepackage{forest}


\usepackage{mdframed}
\usepackage{arabtex}

\usepackage{xcolor}

\usepackage{polyglossia}

\setmainlanguage{english}
\setotherlanguage{urdu}
\newfontfamily\urdufont[Scale=1.25]{jameel.ttf}

\newcommand{\Q}{\text{\texturdu{یکتا}}} % Use this command to type the new quantifier defined.



% Header and footer.
\pagestyle{headandfoot}
\runningheadrule
\runningfootrule
\runningheader{CS/MATH 113, SPRING 2025}{WC 03: Predicate logic and Quantifiers}{\theauthor}
\runningfooter{}{Page \thepage\ of \numpages}{}
\firstpageheader{}{}{}

% \printanswers %Uncomment this line

\title{Weekly Challenge 03: Predicate logic and Quantifiers}
\author{Blingblong} % <=== replace with your student ID, e.g. xy012345
\date{CS/MATH 113 Discrete Mathematics\\Habib University\\Spring 2025}

\qformat{{\large\bf \thequestion. \thequestiontitle}\hfill}
\boxedpoints

\begin{document}
\maketitle



\begin{center}
    \texturdu{اسے کون دیکھ سکتا کہ یگانہ ہے وہ یکتا}
    \\\texturdu{جو دوئی کی بو بھی ہوتی تو کہیں دو چار ہوتا}
    \begin{flushleft}
        \hspace*{4.5cm} (\texturdu{مرزا غالب })
    \end{flushleft}
\end{center}
\begin{questions}
  
    \titledquestion{Kitnay elements thay?} We have already studied two basic quantifiers in this course so far. For a predicate $P$ with some domain $U$, the existential quantifier $\exists x P(x)$ would mean that there are some $x$ in the domain $D$ that satisfies $P$. But a lot of times ``some'' doesn't suit our need. We might need a more specific quantifier. The $\Q_n$ quantifier is is defined as follows; for a predicate $P(x)$ and some domain $U$, $\Q_n x P(x)$ is true if there are exactly $n$ elements in $U$ for which $P$ is true. As logician/mathematician/computer scientists we need some basic quantifier through which we can define all other quantifiers. You being a smart Discrete math student conjecture that you can define $\Q_n$ quantifier by just using existential quantifier or with just using universal quantifier.

    \begin{parts}
        \part Define the $\Q_n$ quantifier by just using the existential quantifier.
        \begin{solution}
            % Add your solution here
        \end{solution}

        \part Define the $\Q_n$ quantifier by just using the universal quantifier.
        \begin{solution}
            % Add your solution here
        \end{solution}
    \end{parts}

      
\end{questions}
\end{document}

%%% Local Variables:
%%% mode: latex
%%% TeX-master: t
%%% End:
